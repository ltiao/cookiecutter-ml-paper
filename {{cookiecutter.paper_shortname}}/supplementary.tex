\documentclass{article}

% if you need to pass options to natbib, use, e.g.:
%     \PassOptionsToPackage{numbers, compress}{natbib}
% before loading neurips_2020
\PassOptionsToPackage{sort,compress}{natbib}

% ready for submission
% \usepackage{neurips_2020}

% to compile a preprint version, e.g., for submission to arXiv, add add the
% [preprint] option:
%     \usepackage[preprint]{neurips_2020}

% to compile a camera-ready version, add the [final] option, e.g.:
%     \usepackage[final]{neurips_2020}

% to avoid loading the natbib package, add option nonatbib:
     \usepackage[nonatbib]{neurips_2020}

\usepackage[utf8]{inputenc} % allow utf-8 input
\usepackage[T1]{fontenc}    % use 8-bit T1 fonts

\usepackage[usenames,dvipsnames]{xcolor}

\usepackage{xr-hyper}
\usepackage[colorlinks]{hyperref}       % hyperlinks
\hypersetup{citecolor=NavyBlue}
\hypersetup{linkcolor=NavyBlue}
\hypersetup{urlcolor=NavyBlue}
\usepackage{url}            % simple URL typesetting

\usepackage{booktabs}       % professional-quality tables
\usepackage{amsfonts}       % blackboard math symbols
\usepackage{nicefrac}       % compact symbols for 1/2, etc.
\usepackage{microtype}      % microtypography

\usepackage[acronym,smallcaps]{glossaries}
\glsdisablehyper

\usepackage{cleveref}

\newacronym{SGD}{sgd}{Stochastic Gradient Descent}

\documentclass{article}

% if you need to pass options to natbib, use, e.g.:
%     \PassOptionsToPackage{numbers, compress}{natbib}
% before loading neurips_2020
\PassOptionsToPackage{sort,compress}{natbib}

% ready for submission
% \usepackage{neurips_2020}

% to compile a preprint version, e.g., for submission to arXiv, add add the
% [preprint] option:
%     \usepackage[preprint]{neurips_2020}

% to compile a camera-ready version, add the [final] option, e.g.:
%     \usepackage[final]{neurips_2020}

% to avoid loading the natbib package, add option nonatbib:
     \usepackage[nonatbib]{neurips_2020}

\usepackage[utf8]{inputenc} % allow utf-8 input
\usepackage[T1]{fontenc}    % use 8-bit T1 fonts

\usepackage[usenames,dvipsnames]{xcolor}

\usepackage{xr-hyper}
\usepackage[colorlinks]{hyperref}       % hyperlinks
\hypersetup{citecolor=NavyBlue}
\hypersetup{linkcolor=NavyBlue}
\hypersetup{urlcolor=NavyBlue}
\usepackage{url}            % simple URL typesetting

\usepackage{booktabs}       % professional-quality tables
\usepackage{amsfonts}       % blackboard math symbols
\usepackage{nicefrac}       % compact symbols for 1/2, etc.
\usepackage{microtype}      % microtypography

\usepackage[acronym,smallcaps]{glossaries}
\glsdisablehyper

\usepackage{cleveref}

\newacronym{SGD}{sgd}{Stochastic Gradient Descent}

\documentclass{article}

% if you need to pass options to natbib, use, e.g.:
%     \PassOptionsToPackage{numbers, compress}{natbib}
% before loading neurips_2020
\PassOptionsToPackage{sort,compress}{natbib}

% ready for submission
% \usepackage{neurips_2020}

% to compile a preprint version, e.g., for submission to arXiv, add add the
% [preprint] option:
%     \usepackage[preprint]{neurips_2020}

% to compile a camera-ready version, add the [final] option, e.g.:
%     \usepackage[final]{neurips_2020}

% to avoid loading the natbib package, add option nonatbib:
     \usepackage[nonatbib]{neurips_2020}

\usepackage[utf8]{inputenc} % allow utf-8 input
\usepackage[T1]{fontenc}    % use 8-bit T1 fonts

\usepackage[usenames,dvipsnames]{xcolor}

\usepackage{xr-hyper}
\usepackage[colorlinks]{hyperref}       % hyperlinks
\hypersetup{citecolor=NavyBlue}
\hypersetup{linkcolor=NavyBlue}
\hypersetup{urlcolor=NavyBlue}
\usepackage{url}            % simple URL typesetting

\usepackage{booktabs}       % professional-quality tables
\usepackage{amsfonts}       % blackboard math symbols
\usepackage{nicefrac}       % compact symbols for 1/2, etc.
\usepackage{microtype}      % microtypography

\usepackage[acronym,smallcaps]{glossaries}
\glsdisablehyper

\usepackage{cleveref}

\newacronym{SGD}{sgd}{Stochastic Gradient Descent}

\documentclass{article}

% if you need to pass options to natbib, use, e.g.:
%     \PassOptionsToPackage{numbers, compress}{natbib}
% before loading neurips_2020
\PassOptionsToPackage{sort,compress}{natbib}

% ready for submission
% \usepackage{neurips_2020}

% to compile a preprint version, e.g., for submission to arXiv, add add the
% [preprint] option:
%     \usepackage[preprint]{neurips_2020}

% to compile a camera-ready version, add the [final] option, e.g.:
%     \usepackage[final]{neurips_2020}

% to avoid loading the natbib package, add option nonatbib:
     \usepackage[nonatbib]{neurips_2020}

\input{preamble/base}
\input{preamble/acronyms}
\input{preamble/supplementary}

\title{
  {{cookiecutter.paper_title}} \\
  \vspace{5pt}
  \it
  Supplementary Material
}

% The \author macro works with any number of authors. There are two commands
% used to separate the names and addresses of multiple authors: \And and \AND.
%
% Using \And between authors leaves it to LaTeX to determine where to break the
% lines. Using \AND forces a line break at that point. So, if LaTeX puts 3 of 4
% authors names on the first line, and the last on the second line, try using
% \AND instead of \And before the third author name.

\author{

  {{author_full_name}}
  
    \thanks{
      {{author_details.info}}
    }
   \\
  {{author_details.address}} \\
  \texttt{
    {{author_details.email}}
  }
  
    {{ loop.cycle('\And', '\AND') }}
  

}

\begin{document}

\maketitle

\begin{abstract}
  We supplement the main paper with further details.
\end{abstract}

\section{Overview}
\label{sec:overview}

It is possible to refer back to \crefmain{sub:style}.

\section{Proof of Some Claim}
\label{sec:proof_of_some_claim}

\section{Details of Experimental Set-Up}
\label{sec:experimental_set_up}

\subsection{Experiments on Dataset A}
\label{sub:experiments_on_dataset_a}

\small

\nocite{*}

\bibliographystyle{plain}
\bibliography{main}

\end{document}


\title{
  {{cookiecutter.paper_title}} \\
  \vspace{5pt}
  \it
  Supplementary Material
}

% The \author macro works with any number of authors. There are two commands
% used to separate the names and addresses of multiple authors: \And and \AND.
%
% Using \And between authors leaves it to LaTeX to determine where to break the
% lines. Using \AND forces a line break at that point. So, if LaTeX puts 3 of 4
% authors names on the first line, and the last on the second line, try using
% \AND instead of \And before the third author name.

\author{

  {{author_full_name}}
  
    \thanks{
      {{author_details.info}}
    }
   \\
  {{author_details.address}} \\
  \texttt{
    {{author_details.email}}
  }
  
    {{ loop.cycle('\And', '\AND') }}
  

}

\begin{document}

\maketitle

\begin{abstract}
  We supplement the main paper with further details.
\end{abstract}

\section{Overview}
\label{sec:overview}

It is possible to refer back to \crefmain{sub:style}.

\section{Proof of Some Claim}
\label{sec:proof_of_some_claim}

\section{Details of Experimental Set-Up}
\label{sec:experimental_set_up}

\subsection{Experiments on Dataset A}
\label{sub:experiments_on_dataset_a}

\small

\nocite{*}

\bibliographystyle{plain}
\bibliography{main}

\end{document}


\title{
  {{cookiecutter.paper_title}} \\
  \vspace{5pt}
  \it
  Supplementary Material
}

% The \author macro works with any number of authors. There are two commands
% used to separate the names and addresses of multiple authors: \And and \AND.
%
% Using \And between authors leaves it to LaTeX to determine where to break the
% lines. Using \AND forces a line break at that point. So, if LaTeX puts 3 of 4
% authors names on the first line, and the last on the second line, try using
% \AND instead of \And before the third author name.

\author{

  {{author_full_name}}
  
    \thanks{
      {{author_details.info}}
    }
   \\
  {{author_details.address}} \\
  \texttt{
    {{author_details.email}}
  }
  
    {{ loop.cycle('\And', '\AND') }}
  

}

\begin{document}

\maketitle

\begin{abstract}
  We supplement the main paper with further details.
\end{abstract}

\section{Overview}
\label{sec:overview}

It is possible to refer back to \crefmain{sub:style}.

\section{Proof of Some Claim}
\label{sec:proof_of_some_claim}

\section{Details of Experimental Set-Up}
\label{sec:experimental_set_up}

\subsection{Experiments on Dataset A}
\label{sub:experiments_on_dataset_a}

\small

\nocite{*}

\bibliographystyle{plain}
\bibliography{main}

\end{document}


\title{
  {{cookiecutter.paper_title}} \\
  \vspace{5pt}
  \it
  Supplementary Material
}

% The \author macro works with any number of authors. There are two commands
% used to separate the names and addresses of multiple authors: \And and \AND.
%
% Using \And between authors leaves it to LaTeX to determine where to break the
% lines. Using \AND forces a line break at that point. So, if LaTeX puts 3 of 4
% authors names on the first line, and the last on the second line, try using
% \AND instead of \And before the third author name.

\author{

  {{author_full_name}}
  
    \thanks{
      {{author_details.info}}
    }
   \\
  {{author_details.address}} \\
  \texttt{
    {{author_details.email}}
  }
  
    {{ loop.cycle('\And', '\AND') }}
  

}

\begin{document}

\maketitle

\begin{abstract}
  We supplement the main paper with further details.
\end{abstract}

\section{Overview}
\label{sec:overview}

It is possible to refer back to \crefmain{sub:style}.

\section{Proof of Some Claim}
\label{sec:proof_of_some_claim}

\section{Details of Experimental Set-Up}
\label{sec:experimental_set_up}

\subsection{Experiments on Dataset A}
\label{sub:experiments_on_dataset_a}

\small

\nocite{*}

\bibliographystyle{plain}
\bibliography{main}

\end{document}
